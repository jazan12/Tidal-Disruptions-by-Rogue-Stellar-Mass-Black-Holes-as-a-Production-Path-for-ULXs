\section{Discussion}\label{section:conc}

\begin{itemize}
\item Why is this attractive - summarize that we know of  many stellar-mass BH sources with masses up to $\sim20$
\item Extreme objects like HLX--1 unlikely to be explained by this (Farrell+2009, Davis+2011)
\end{itemize}



\subsection{Relevance to periodic signals in blazar jets}


\subsection{``Isolated'' Black Hole X-ray Outbursts from TDE fallback disks} 


\subsection{Critical Density of the Accretion Disk required for catastrophic inspiral}


\subsection{Relationship between Mplanet,  the initial planet-BH separation (aka impact param),  BH-star distance, and host-star Mass (for a given wind model)}


\subsection{Input Bruzual and Charlot or some sort of IMF to estimate rates given a binary IMF and associated binary fraction}


\subsection{Maybe simply look up binary fraction and assume that half of the ones over the critical mass limit will produce BHs as a starting point}


\subsection{Remember, the calculation is to be done over the age of the universe - which means want the total number of those stars formed, even if most go boom quickly} 


\subsection{Interesting to explore interactions between the disk and
the planet.  Possible for resonant modes excited in the disk by the
planet to either excite or damp planet interaction? Possible warps
would help align; keep in mind how close they would be.  Tidal
forces on the outskirts of disk being dragged on by the planet could
very possibly lead to excitation and destabalization of disk or
planet or both.  Likely a strong function of the wind from the star
and the mass of the disk, distance of disk to planet at periastron,
etc.}
