\section{Introduction}\label{section:Intro}


The nature of ultraluminous X-ray sources (ULXs) -- off-nuclear point
sources with $L>10^{39}$erg/s -- has been the subject of a
long-standing debate: Are ULXs comprised of stellar-mass or
intermediate-mass ($10^{2}-10^4~\msun$) BHs?  If stellar-mass, then
the BH population must be accreting at sustained super-Eddington
rates.  On the other hand, the intermediate-mass scenario requires an
abundant population of BHs for which there is, as of yet, no clear
dynamical evidence (though see the promising recent result from
\citealt{Pasham_2014}).

In forthcoming work, I propose a formation model for ULXs wherein
these systems arise naturally from tidal-disruption encounters between
red giants and the known population of stellar-mass BHs (Steiner et
al., in prep.).  The ULXs themselves result from the decaying
accretion phase of such encounters.  Specifically, the tidal
disruption event (TDE) fuels the BH at super-Eddington rates for
several thousand years, during which time the system appears as a ULX;
beyond this timeframe, the accretion rate declines to sub-Eddington
levels, and the system would then blend in with the population of
``usual'' BH X-ray binaries.

This model is successful in several respects. It explains the
association of ULXs with star forming regions, while also accounting
for the discovery of ULXs in old stellar populations (Fig.~2).  The
lack of orbital signatures for many ULX systems is understood, in this
paradigm, because they are companionless.  Unfortunately, the ULX rate
estimate for this scenario sufficiently tunable that it alone is not
discriminating on this model's viability.  However, there is one clear
and unambiguous signature of this model: the ULX population should
decline in intensity over time, at a level of several percent per
decade.

To check for this smoking-gun validation, we have garnered all ULX
systems from the \citet{Walton_2011} XMM-Newton ULX catalog for which
there are multiple observations.  In total, this consists of 529
measurements of 115 sources.  Unfortunately, the large {\em intrinsic}
variation in ULX systems dilutes the signal of any secular time
evolution.  Specifically, the data indicate that intrinsic variability
occurs at a level of $\sim0.4$ dex.  While the best fit yields a
negative slope, consistent with the expected decline in flux, the
gross variability weakens the net signal; the end determination is
just $1\sigma$ from a null result.  In short, more data is required.


The tidal disruptions of flyby stars passing close to supermassive
black holes in the centers of galaxies produce a characteristic 
flare and decay event.  Recently, Swift 1644...
These flares the properties of ... and 



XX
The key property of the tidal disruption event is the 

The recent identification of ULX NS is consistent 

Also discuss the TZO in light of this
XX


The disruption of smaller asteroid-to-planet sized objects has
meanwhile been proposed as a possible explanation for the flaring
activity of the Milky Way's supermassive black hole, Sgr A* \citep{SSS}.


We consider close encounters between stellar-mass black holes and giants, whose tidal radii are large targets for passing stellar-mass black holes.  In fact, for a normal stellar population these encounters dominate the tidal disruption cross section for stars interacting with passing black holes.  As we describe, the result of such a collision is a decaying X-ray transient, with a very long lifetime.  Neglecting any feedback processes which may regulate or otherwise alter the oncoming accretion flow, one expects a declining X-ray star emitting at super-Eddington rates for roughly $10^4$ years.  As the source fades gradually, it will transition from a so-called "ultraluminous" X-ray source and merge amongst the population of "X-ray binaries", where it is likely to appear quite like any other transient object. 

Several distinguishing characteristics include the lack of a companion star, and accordingly, a larger accretion disk, which may be more prone to dynamical instabilities and warps, especially given the range of specific angular momentum resulting from the dispersion of the disrupted stellar gas.

One consequence of having a population of disruption remnants manifesting as ULXs is that one expects a decline in ULX intensity over time.  Of course, empirically speaking, ULX sources are evidently often highly variable.  It is therefore challenging to unambiguously assess the behaviour of the population by looking to a single or mere handful of sources.  As a test of this proposition, we have tested one of the largest archives of ULX systems, the XMM-Newton ULX catalog of \citep{Walton_2012}.  From that catalog, we isolate the 115**?** sources with multiple detections.  Those sources are fitted to a model which describes their luminosity as a global power-law, whose index is the quantity of interest.  Our model predicts that the slope should be slighly negative; whereas the null hypothesis is that the sources should be constant.   This measurement is complicated by the aforementioned large degree of intrinsic variability in the ULX population.  If this stochastic noise were absent, it would be possible to clearly rule for or against our model.  However, although the best fit to the data favor a negative slope, the large variability, which we incorporate as a noise term to bring the reduced chisq to unity is so large that the difference from zero slope is marginal.  However, because the timesecale .... propose for more time....







This model has the benefit of naturally accounting for the observed population and density of ULXs, and naturally joins the luminosity function of ULXs, HLXs, and XRBs.  One obstacle to a more concrete set of predictions is the relatively uncertain domain of hypercritical accretion.  Given a model which relates $\mdot$ and $L$ at superEddington rates, and, less important, the distribution of black hole and stellar masses, it is straightforward to derive an expectation for the ULX luminosity function.  However, there is presently much uncertainty in the ...SUPEREDD disc.... point to work by Olek and crew which may help.















We show that planets which may exist around the companion stars in
black-hole binary systems, through a variety of scenarios, will come
into close proximity with the central black hole and be sheared apart.
This process and the resulting emission are in many ways analagous to
the tidal-disruption events observed around supermassive black holes,
but with evolution on typically much shorter timescales.  Owing to the
fluence ratio and the ratio of accreted masses between the proposed
events to GRBs ($\sim 10^{-6}$) we refer to these events as $\mu$GRBs.


Discuss : 
\begin{enumerate}
\item  Intro  
\item  General picture  
\begin{itemize}
\item (Rates - maybe move into its own section?) 
\item  Scenario I   
\item  Scenario II  
\end{itemize}
\item  X-ray and Gamma-ray Emission [Sasha] 
\item  The Experience of the Planet [Robert] 
\item  Candidate Objects (``Tentative candidates'') 
\item  Concs  
\end{enumerate}


We describe our formalism in Section~\ref{section:general} and we
explore several different capture channels and estimate their rates in
Section~\ref{section:rates}.  The response of the accretion disk and a
sketch of the high-energy emission are described in
Section~\ref{section:emission}.  The heating and disruption of the
planet are explored in Section~\ref{section:planet}.  Finally, we
suggest several candidate events which may be explainable via this
mechanism (Section~\ref{section:candidates}) and conclude with a
summary.
