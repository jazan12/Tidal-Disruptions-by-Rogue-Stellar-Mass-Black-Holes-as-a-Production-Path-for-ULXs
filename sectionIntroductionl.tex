\section{Introduction}\label{section:Intro}


The tidal disruptions of flyby stars passing close to supermassive
black holes in the centers of galaxies produce a characteristic 
flare and decay event.  Recently, Swift 1644...
These flares the properties of ... and 



XX
The key property of the tidal disruption event is the 

The recent identification of ULX NS is consistent 
XX



The disruption of smaller asteroid-to-planet sized objects has
meanwhile been proposed as a possible explanation for the flaring
activity of the Milky Way's supermassive black hole, Sgr A* \citep{SSS}.


We show that planets which may exist around the companion stars in
black-hole binary systems, through a variety of scenarios, will come
into close proximity with the central black hole and be sheared apart.
This process and the resulting emission are in many ways analagous to
the tidal-disruption events observed around supermassive black holes,
but with evolution on typically much shorter timescales.  Owing to the
fluence ratio and the ratio of accreted masses between the proposed
events to GRBs ($\sim 10^{-6}$) we refer to these events as $\mu$GRBs.


Discuss : 
\begin{enumerate}
\item  Intro  
\item  General picture  
\begin{itemize}
\item (Rates - maybe move into its own section?) 
\item  Scenario I   
\item  Scenario II  
\end{itemize}
\item  X-ray and Gamma-ray Emission [Sasha] 
\item  The Experience of the Planet [Robert] 
\item  Candidate Objects (``Tentative candidates'') 
\item  Concs  
\end{enumerate}


We describe our formalism in Section~\ref{section:general} and we
explore several different capture channels and estimate their rates in
Section~\ref{section:rates}.  The response of the accretion disk and a
sketch of the high-energy emission are described in
Section~\ref{section:emission}.  The heating and disruption of the
planet are explored in Section~\ref{section:planet}.  Finally, we
suggest several candidate events which may be explainable via this
mechanism (Section~\ref{section:candidates}) and conclude with a
summary.
