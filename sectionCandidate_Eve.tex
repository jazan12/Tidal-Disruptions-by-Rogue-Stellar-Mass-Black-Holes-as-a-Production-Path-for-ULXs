\section{Candidate Events}\label{section:candidates}


{\em Look to High-Energy Transients by Gehrels \& Cannizzo (2012).  }


{\bf XRO 080109 (SN 2008D) ---} {\it On 2008 January 9 \emph{Swift}/XRT
serendipitously discovered an extremely bright X-ray transient
(Soderberg et al. 2008) while undertaking a preplanned observation
of the galaxy NGC 2770 ($d=27$ Mpc).  Two days earlier
source.  X-ray outburst (XRO) 080109 lasted about 400 s and occurred
in one of the galaxy's spiral arms.  XRO 080109 was not a GRB (no
$\gamma-$rays were detected), and the total X-ray energy $E_X \simeq
2\times 10^{46}$ erg was orders of magnitude lower than a GRB.  The
peak luminosity $\sim$$6\times10^{43}$ erg s$^{-1}$ is much greater
than the Eddington luminosity for a $\sim$$1\msun$ object, and also
from type I X-ray bursts. Therefore the standard accretion and
thermonuclear flash scenarios are excluded.


Simultaneous \emph{Swift}/UVOT 
observations did not reveal a counterpart,
but UVOT observations at 1.4 hr showed a brightening.
Gemini North 8-m telescope 
observations beginning at 1.7 d 
revealed a spectrum   suggestive
of a young SN (Soderberg et al. 2008).
Later observations
confirmed the spectral features.
The transient was classified as
a type Ibc SN based on the lack of H, and weak Si features.


Soderberg et al. (2008) argue that the X-ray flash (Figure 9)
indicates a trans-relativistic shock breakout from a SN, where the
radius at breakout is $\ga7\times 10^{11}$ cm, and the shock velocity
at breakout is $\gamma\beta\la1.1$.  Soderberg et al. (2008) estimate
a circumstellar density which yields an inferred pre-SN mass loss
rate $\sim10^{-5}\msun {\rm yr}^{-1}$, reinforcing the notion of a Wolf-Rayet
progenitor.  The similarity between the shock break-out properties of
the He-rich SN 2008D and the He-poor GRB-associated SN 2006aj are
consistent with a dense stellar wind around a compact Wolf-Rayet
progenitor.


X-ray and radio observations presented by Soderberg et al. (2008) of
SN 2008D are the earliest ever obtained for a normal type Ibc SN.  At
$t < 10$ d, the X-ray and peak radio luminosities are orders of
magnitude less than those of GRB afterglows (Berger et al. 2003a,
Frail et al. 2003), but comparable to those of normal type Ibc SN
(Berger et al. 2003b, Kouveliotou et al. 2004).
}\\


{\bf GRB 101225 (the ``Christmas Burst'') ---} {\it GRB 101225 was quite
unusual: it had a $T_{90} > 1700$ s and exhibited a curving decay
when plotted in the traditional $\log F - \log t$ coordinates.  The
total BAT fluence was $\ga3\times 10^{-6}$ erg cm$^{-2}$.  The XRT
and UVOT found a bright, long-lasting counterpart.  Ground based
telescopes followed the event, mainly in $R$ and $I$, and failed to
detect any spectral features.  At later times a color change from
blue to red was seen; \emph{HST} observations at 20 d found a very
red object with no apparent host.  Observations from the Spanish
Gran Telescopio Canarias at 180 d detected GRB 101225 at $g_{\rm
AB}= 27.21\pm0.27$ mag and $r_{\rm AB}= 26.90\pm0.14$ (Th\"one et
al. 2011).  Considered together, these characteristics are unique to
this burst (Figure 12), and led Campana et al. (2011) to propose
that it was caused by a minor body like an asteroid or comet
becoming disrupted and accreted by a NS.  Depending on its
composition, the tidal disruption radius would be $\sim10^5 - 10^6$
km.  Campana et al. find an adequate fit to the light curve by
positing a $\sim 5\times 10^{20}$ g asteroid with a periastron
radius $\sim9000$ km.  If half the asteroid mass is accreted they
derive a total fluence $4\times 10^{-5}$ erg cm$^{-2}$ and distance
$\sim3$ kpc.


Th\"one et al. (2011) offer a different explanation for GRB 101225
$-$ the merger of a He star and a NS leading to a concomitant SN.
They derive a pseudo-redshift $z=0.33$ by fitting the
spectral-energy distribution and light curve of the optical emission
with a GRB- SN template.  Thus in their interpretation the event was
much more distant and energetic.  They argue for the presence of a
faint, unresolved galaxy in deep optical observations, and fit the
long term light curve with a template of the broad-line type Ic SN
1998bw associated with GRB 980425.  If their distance is correct as
well as their interpretation of a component emerging at 10 d as
being a SN, then its absolute peak magnitude $M_{\rm V, \ abs} =
-16.7$ mag would make it the faintest SN associated with a long GRB.
The isotropic-equivalent energy release at $z=0.33$ would be
$>1.4\times 10^{51}$ erg which is typical of other long GRBs but
greater than most other low-redshift GRBs associated with SNe.  }


TABLE idea:
make a list of planet masses, either tidal capture, or close passage, and present the timescales, energy fluences, event rates (beaming and 
nonbeaming both?)
