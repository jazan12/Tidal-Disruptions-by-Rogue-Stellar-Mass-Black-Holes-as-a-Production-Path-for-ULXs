{\bf Tidal Disruptions by Rogue Stellar-Mass Black Holes:  Implications for GRB demographics and ULXs}\\



  We propose a mechanism of producing Gamma-Ray Bursts.  These events, caused by the disruption of a white dwarf around a stellar-mass black hole, occur over timescales of several seconds, with an energy release and collimation both expected to be slightly below that of a collapsar.  The frequency at which such disruptions occurs should be $\sim1$ per galaxy per Hubble time from stellar collisions.  Binary evolution in which close white dwarf -- black hole binary systems are produced may amplify the rates.  These bursts should be observable within the visible universe; we roughly estimate an occurence rate of 100/yr, which is comparable to the rate of short GRBs.  We propose this mechanism as a subset -- possibly an alternative scenario -- for the generation of short GRBs.  With the proposed mechanism, it is quite natural for these events to trace the old stellar populaion, rather than recent star formation.  This is an observed characteristic of short GRBs. 
  
    with a typical
  fluence of order $10^{46}$~erg.  These transients can be produced
  through a variety of dynamical or tidal capture-type interactions
  between the black hole and a planet orbiting its companion star.  We
  discuss two cases of chief interest, one in which the planet is
  captured in a retrograde orbit about the black hole and dissipates
  angular momentum in the disk during a secular inspiral; the other
  involves fast-acting tidal disruption which destroys a planet after
  an external interaction perturbs the binary orbit.
  
  
  \keywords{accretion, accretion discs --- black hole physics --- X-rays:
  binaries}
