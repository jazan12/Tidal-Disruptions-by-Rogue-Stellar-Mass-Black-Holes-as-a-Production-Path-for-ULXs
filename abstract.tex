%{\bf Tidal Disruptions by Rogue Stellar-Mass Black Holes:  Implications for GRB demographics and ULXs}\\

We propose that some of the observed population of ultraluminous X-ray sources (ULXs) are formed by the tidal 
disruption of post-main sequence stars by the known population of stellar-mass black holes.  This mechanism is especially productive in star clusters.  Specifically, we consider a parent population of black hole / giant tidal encounters which arise naturally via dynamics.  Of primarily interest is the aftermath of a disruption event, as the system declines from a highly super-Eddington mass accretion rate to the sub-Eddington regime.  During this long-lived transitional phase, expected to last thousands of years, the accretion of the giant remnant onto the stellar-mass black hole would manifest as a ULX.   Implications and testable consequences of this scenario are discussed.

% We propose the tidal disruption of post main-sequence stars by stellar-mass black holes as a candiate mechanism for
% producing the observed population --or a substantial fraction thereof-- of ultra-luminous X-ray sources (ULX).   Specifically, we consider a parent population of such systems which naturally occurs through dynamical encounters.  The short-lived transient event immediately following a major disruption is highly energetic and produces a signature at XXX.  However, we primarily consider the aftermath of a disruption event, during which the source declines from highly super-Eddington mass accretion rates, to the sub-Eddington regime.  During this long-lived transitional phase, lasting thousands of years, the accretion of the giant remnant onto the stellar-mass black hole would manifest as a ULX.   Implications and testable consequences of this scenario are discussed.


{\bf ULX, TZ object, etc.}

  \keywords{accretion, accretion discs --- black hole physics --- stars: kinematics and dynamics}
