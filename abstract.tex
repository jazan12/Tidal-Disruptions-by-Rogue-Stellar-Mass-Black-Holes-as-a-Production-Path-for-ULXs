{\bf Are Ultraluminous X-ray Sources Powered by Tidally Disrupted Giants? \\ Tidal Disruptions by Rogue Stellar-Mass Black Holes:  Implications for GRB demographics and ULXs}\\

We propose the tidal disruption of post main-sequence stars by stellar-mass black holes as a candiate mechanism for
producing the observed population --or a substantial fraction thereof-- of ultra-luminous X-ray sources (ULXs).   Specifically, we consider a parent population of such systems which naturally occurs through dynamical encounters.  The short-lived transient event immediately following a major disruption is highly energetic and may be 
accompanied by powerful jet emission.  We consider the thousands of years immediately following a disruption event, during which the source declines from highly super-Eddington mass accretion rates down to the sub-Eddington regime.  During this long-lived transitional phase, the accretion of the giant remnant onto the stellar-mass black hole manifests as a ULX.   We discuss the implications and testable consequences of this scenario, and consider its viability for the known ULX population.


{\bf ULX, TZ object, etc.}

  \keywords{accretion, accretion discs --- black hole physics --- stars: kinematics and dynamics}
