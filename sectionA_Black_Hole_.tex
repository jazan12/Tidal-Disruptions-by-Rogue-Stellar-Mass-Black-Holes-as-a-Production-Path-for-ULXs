\section{Tidal Disruption of a Star by a Stellar-Mass Black Hole}\label{section:general}


For simplicity, we begin considering a constant density star (mass $M_*$, radius $R_*$).  The star will be destroyed if its pericenter passage $R_p < R_t$, wherer

\begin{equation}
R_t = 9 \times 10^{-3} {\rm AU}~ r_{*} m_{*}^{-1/3} m_{BH,7}^{1/3}
\end{equation}

and $m_*$, $r_*$ are the mass and radius, respectively, in Solar units, while $m_{{\rm BH},7} = M_{BH}/(7~\Msun)$.  Any tidal encounter from a rogue BH should occur in the ``parabolic" regime of TDEs.   

\begin{equation}
t_{\rm fall} \approx 9 \times 10^3~{\rm s}~m_{{\rm BH},7}^{1/2}m_{*}^{-1}r_{*}^{3/2}\\
t_{\rm fall} \approx 0.3~{\rm yr}~m_{{\rm BH},7}^{1/2}m_{*}^{-1}\left(\frac{r_{*}}{100}\right)^{3/2}
\end{equation}

After the initial disruption, the mass fallback will roughly decline following a $t^{-5/3}$ power law \citep{Rees_1988}. At the onset of fallback, and for a long time thereafter, accretion will be highly super-Eddington, with a fallback rate

\begin{equation}
\frac{\dot{M}}{\dot{M_{\rm Edd}}} \approx 8 \times 10^{6} \left(\frac{\eta}{0.1}\right)m_{\rm BH,7}^{-3/2}m_{*}^{2}r_*^{-3/2} \left(\frac{t+t_{\rm fall}}{t_{\rm fall}}\right)^{-5/3}; \qquad t > 0 \\
\frac{\dot{M}}{\dot{M_{\rm Edd}}} \approx 8 \times 10^{9} \left(\frac{\eta}{0.1}\right)m_{\rm BH,7}^{-3/2}m_{*}^{2}\left(\frac{r_*}{100}\right)^{-3/2} \left(\frac{t+t_{\rm fall}}{t_{\rm fall}}\right)^{-5/3}; \qquad t > 0
\end{equation}

where $t=0$ marks the onset of the disruption event, and $0 < \eta < 1$ is the efficiency with which the accretion disk radiates away rest-mass energy.  We begin by making the simplifying assumption that the fallback mass is efficiently accreted onto the BH ({\em note that this is different from assuming that the accretion is ratiatively efficient}), i.e., we assume that mass loss and radiative or mechanical feedback can be neglected.  Our simplistic prediction is then that the fallback gas will supply the BH at super-Eddington rates for several thousand years (giant TDE) down to hundreds of years (dwarf TDE).   

At the same time, we note that ULXs are empirically known to often exhibit extreme variability.  It is reasonable to expect that our simplifying assumption is flawed to some degree.  The overall effect of mass loss in the flow e.g., in an ADIOS \citep{Begelman??} or similar flow, would be to reduce the flow and shorten the accretion timescales.  However, the effect of feedback is subtler and more nuanced.  Feedback may regulate the nozzle speed of the fallback gas stream and thereby prolongue the super-Eddington phase, or if strong enough, may sporradically quench the superEddington flow.  The specifics of this interaction are beyond the scope of this work.  For simplicity, we aim to capture this complication in a single ``fudge-factor'' variable $f$, which represents the combined effect of outflows and feedback on the sueprEddington timescale, the attribute of strongest importance to this work which is affected.  

 More detailed calculations of accretion disks in fallback-fueled TDE systems have shown XXXX \citep{Shen_2014,Coughlin_2014}.  *** 

In particular, the accretion can be substantially modified, so that the flow is suppressed at early times as the disk ... prolongued later .... 







\bf{Outline and Thoughts:}\\
\begin{itemize}
%\item First, describe the overall mechanics of TDEs, and give Nick's calculations, etc. \\
\item What stars are likely to be disrupted?  (James <-- look into the net cross section, MESA for Sun, get R(t)) (rho(t)) \\
\item Given the dominant signal, etc., how quickly will the signal from stars decline, how fast for giants? .... (Red clump / HB vs RGB ) ?  James on index, ... \\ 
\item Next, where (clusters, dense environments, etc.), and rate calculations, figure like from Rosanne \\
\item Discussion piece: 
\begin{enumerate}
    \item predictions for overall decline (calculate index) ... 
    \item luminosity function (how to relate mdot to L ...) describe, predict  ... joining on to LMXB .. number density ... 
    \item {\it sensitivity of all above to initial density},  
    \item where are we likely to find ULXs? .... 
    \item comparison with NSs.... what does it mean that one ULX is a NS .... 
    \item TZO ... focusing on TDEs 
    \end{enumerate}
    
\item Summary & Concs: outlined likely channel which results in .... consequences and testability ... 
\end{itemize}






Each galaxy is thought to have $\approx 10^X$ stellar-mass black
holes, which are born in the supernova deaths of high-mass progenitor
stars.  Although these objects are so numerous, we have only detected
$\approx$40 such black holes in the Milky Way because to be detected
they must undergo an X-ray outburst which usually requires that they
reside in close binary systems.  To date, no stellar-mass black hole
in our Galaxy has been otherwise detected and observed.


Yet even while those few close binary systems light up the X-ray sky
during their decadal outbursts, there should be many more systems in
which the binary is not as tightly bound, and the black hole and
companion star may be separated by several AU.  As a consequence, a
usual companion star would then not fill its Roche Lobe, and so
accretion onto the black hole would be limited to a weak capture of
wind from the companion star.  Furthermore, since we have a clearer
understanding now than ever before that planets are abundant in
stellar systems, some of these wide binary systems with black hole
primaries must also be host to one or several planets.  


From virial considerations, it is less likely that a planet would
survive in orbit about a star that lost most of its mass in a
supernova explosion, and so rather than envisioning a system with a
planet in direct orbit around a black hole, we begin with a system in
which the black hole has shed any planets but the companion star may
retain its own.


There are many situations in which the planet will uneventfully orbit
the companion star or companion-star, black hole system.  For now, we
put aside the question of the probability of a capture of the planet
by the black hole; we will revisit this later while estimating the
rate of such events.  We now suppose the unusual case in which the
planet is in some way disturbed and has a close passage to the black
hole.  In particular, we adopt two scenarios: (1) the planet has
experienced a strong kick and approaches the black hole approximately
radially in a hyperbolic or parabolic orbit; (2) the planet is gently
exchanged between companion star and black hole, where it encounters
an accretion disk.


\subsection{Other Formation Channels}
While we leave a detailed investigation of alternate channels to other works, 
there are a number of alternate avenues by which such disruptions may also be induced.

Foremost is via binary evolution.  Certainly, binary systems are the focus of most other
models of ULX production, and they may be just as relevant here.  Because the signal in question is dominated by giants and massive stars, our expectation is that the natal kicks which may be imparted at the time of formation of a stellar-mass BH would sometimes place it on a ``disruptive" trajectory with its companion star.  The systems of relevance here would mostly be fairly wide binary systems ($a > 100$ AU) in order that the companion survive post-main sequence evolution without entering a common-envelope phase.   The details required for a precise calculation of this cross-section are beyond the scope of this paper.  We merely point out that it is quite plausible for this to be the dominant channel for producing such TDE-induced ULXs.


Also relevant may be contribution from open clusters.  On the one hand, active star formation..., young, etc.  On the other, how long does a BH survive in one (how long to escape?)...{\bf note that BH with typical kick would escape open cluster in ~1e5 yrs}....  Density and number of stars both much reduced compared to the population of GCs.  However, may be important in considering young stellar populations in particular, and star forming environments.



\section{A test: luminosity evolution}

\citet{Khabibullin_2014} have explored a similar question.  They looked for sources which had faded by a factor $\gtrsim 10$/decade to search for TDEs of stars by SMBHs, and conclude based upon a handful of candidate events, that the rate is of order $10^{-5}/yr$ per galaxy.  Our calculations indicate that a similar order of stellar disruptions but by stellar-mass BHs are expected.   

However, unlike a TDE by an AGN, the X-ray emission for the stellar-mass event is likely to persist for a long





Retrograde Orbits the Best\\
Prograde less so
