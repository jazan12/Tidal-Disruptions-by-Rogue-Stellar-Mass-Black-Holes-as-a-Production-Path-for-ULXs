\section{A Black Hole Encounters a Star}\label{section:general}


For simplicity, we begin considering a constant density star (mass $M_*$, radius $R_*$).  The star will be destroyed if its pericenter passage $R_p < R_t$, where

\begin{equation}
R_t = 9 \times 10^{-3} {\rm AU}~ r_{*} m_{*}^{-1/3} m_{BH,7}^{1/3}
\end{equation}

and $m_*$, $r_*$ are the mass and radius, respectively, in Solar units, while $m_{{\rm BH},7} = M_{BH}/(7~\Msun)$.  Any tidal encounter from a rogue BH should occur in the ``parabolic" regime of TDEs.   Meanwhile, the characteristic fallback timescale for the TDE is

\begin{equation}
t_{\rm fall} \approx 9 \times 10^3~{\rm s}~m_{{\rm BH},7}^{1/2}m_{*}^{-1}r_{*}^{3/2}\\
t_{\rm fall} \approx 0.3~{\rm yr}~m_{{\rm BH},7}^{1/2}m_{*}^{-1}(\frac{r_{*}}{100})^{3/2}
\end{equation}



\bf{Outline and Thoughts:}\\
\begin{itemize}
\item First, describe the overall mechanics of TDEs, and give Nick's calculations, etc. \\
\item What stars are likely to be disrupted?  (James <-- look into the net cross section, MESA for Sun, get R(t)) (rho(t)) \\
\item Given the dominant signal, etc., how quickly will the signal from stars decline, how fast for giants? .... (Red clump / HB vs RGB ) ?  James on index, ... \\ 
\item Next, where (clusters, dense environments, etc.), and rate calculations, figure like from Rosanne \\
\item Discussion piece: 
\begin{enumerate}
    \item predictions for overall decline (calculate index) ... 
    \item luminosity function (how to relate mdot to L ...) describe, predict  ... joining on to LMXB .. number density ... 
    \item {\it sensitivity of all above to initial density},  
    \item where are we likely to find ULXs? .... 
    \item comparison with NSs.... what does it mean that one ULX is a NS .... 
    \item TZO ... focusing on TDEs 
    \end{enumerate}
    
\item Summary & Concs: outlined likely channel which results in .... consequences and testability ... 
\end{itemize}






Each galaxy is thought to have $\approx 10^X$ stellar-mass black
holes, which are born in the supernova deaths of high-mass progenitor
stars.  Although these objects are so numerous, we have only detected
$\approx$40 such black holes in the Milky Way because to be detected
they must undergo an X-ray outburst which usually requires that they
reside in close binary systems.  To date, no stellar-mass black hole
in our Galaxy has been otherwise detected and observed.


Yet even while those few close binary systems light up the X-ray sky
during their decadal outbursts, there should be many more systems in
which the binary is not as tightly bound, and the black hole and
companion star may be separated by several AU.  As a consequence, a
usual companion star would then not fill its Roche Lobe, and so
accretion onto the black hole would be limited to a weak capture of
wind from the companion star.  Furthermore, since we have a clearer
understanding now than ever before that planets are abundant in
stellar systems, some of these wide binary systems with black hole
primaries must also be host to one or several planets.  


From virial considerations, it is less likely that a planet would
survive in orbit about a star that lost most of its mass in a
supernova explosion, and so rather than envisioning a system with a
planet in direct orbit around a black hole, we begin with a system in
which the black hole has shed any planets but the companion star may
retain its own.


There are many situations in which the planet will uneventfully orbit
the companion star or companion-star, black hole system.  For now, we
put aside the question of the probability of a capture of the planet
by the black hole; we will revisit this later while estimating the
rate of such events.  We now suppose the unusual case in which the
planet is in some way disturbed and has a close passage to the black
hole.  In particular, we adopt two scenarios: (1) the planet has
experienced a strong kick and approaches the black hole approximately
radially in a hyperbolic or parabolic orbit; (2) the planet is gently
exchanged between companion star and black hole, where it encounters
an accretion disk.


\subsection{Tidal Disruption}
Triggered by a perturbation in e.g., glob clust?


\subsection{Secular Migration}
Retrograde Orbits the Best\\
Prograde less so
