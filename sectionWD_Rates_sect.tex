\section{WD Rates}
section{Rates}


We approximate the population of minor bodies in the WD planetary system by a single distribution in mass.  As shown by \citet{Dohnanyi_1969}, collisional systems such as asteroids in the Solar System, and presumably those about WDs, are distributed as $dN(M) \propto M^{-11/6} dM$ for bodies smaller than a critical, large mass -- $M_*$.  Here, $dN(M)$ refers to the number of objects with mass between $M$ and $M+dM$.  This law most readily applies at small scales, and is robust to many assumptions regarding the microphysics adopted by Dohnanyi (see, e.g, \citealt{Williams_Wetherill_1994,Campo_Bagatin_1994}).  However, near and above $M_*$ (i.e., roughly the size of the largest object likely to collide), a flattening is expected to occur in the distribution, to a value near $dN/dM \propto M^{-5/3}$ \citep{Klacka_1992,Safronov_1979,Anders_1965}.  Written generically as $\frac{dN(M)}{dM} \propto M^{-\beta}$, we allow values between $\beta =5/3$ and $\beta = 11/6$.  Solar System data are in generally good agreement with these estimates (e.g., \citealt{Schlichting_2013,Fuentes_2010}).


Over the last decade, spectroscopic data of white dwarf systems has demonstrated that $\gtrsim 1/3$ of WD systems have substantial ``pollution'' in their atmospheres which results from metal accretion from rocky detritus which is disrupted in close proximity to the WD, sometimes forming a dusty disk (e.g., \citealt{Jura_2003, Jura_2008, Jura_2007, Jura_2007b, Reach_2009, Zuckerman_2010, Zuckerman_2012, Farihi_2012, Farihi_2013,Xu_2013}).


Because our observations are insensitive to timescales much shorter than the metal settling time in a WD atmosphere ($\sim 10^5$ years, depending on the WD temperature and composition \citealt{Koester_2009}), we can only assess the impacts in aggregate.  On this basis, the average mass influx appears to be $\dot{M} = 10^{15}-10^{17}{\rm g/yr}$ \citep{Zuckerman_2010,Barber_2012} over such timescales.  


The rate at which a given mass will scatter into the tidal radius of the WD and undergo disruption is,


\begin{equation}
\frac{dN_{\rm disrupt}(M)}{dt} = f(M) \frac{dN(M)}{dM}.
\end{equation}


where $f(M)$ is the rate at which a body of mass $M$ is scattered into the tidal radius.  One expects $f(M)$ to be approximately constant.  Therefore,
\begin{equation}
\dot{M} = \int_0^{M_*} M f M^{-\beta} dM  = \frac{f}{2-\beta} M_*^{-\beta+2}
\end{equation}


The contribution to this rate from a given mass bin is given by $\frac{d}{dM}(\dot{M})$, equivalently, 
\begin{equation}
\frac{dN_{\rm disrupt}({\rm log}M)}{dt} = f M^{-\beta+1}.  
\end{equation}


Using the measured $\dot{M}$ to solve for $f$, and substituting, 
\begin{equation}
\frac{dN_{\rm disrupt}({\rm log}M)}{dt} = (2-\beta) \frac{\dot{M}}{M_*} \left(\frac{M}{M_*}\right)^{-\beta+1}.
\end{equation}


The fraction of disruption events which will result in a direct collision is quite small, and we can estimate the rate from the relative cross section.  Because scattered, bound objects passing close to the WD take place in the regime of strong gravitational focusing, which effectively enhances the surface cross section (e.g., \citealt{Pineault_Landry_1994}), the proportion of disruptions which are collisions is approximately $R_{\rm WD}/R_{\rm tidal}$. Then,
\begin{equation}
\frac{dN_{\rm coll}({\rm log}M)}{dt} = R_{\rm WD} \left(\frac{\rho(M)}{M_{WD}}\right)^{1/3} (2-\beta)\frac{\dot{M}}{M_*} \left(\frac{M}{M_*}\right)^{-\beta+1}.
\end{equation}
\begin{equation}
\frac{dN_{\rm coll}({\rm log}M)}{dt} \approx 10^{-2} (2-\beta)\frac{\dot{M}}{M_*} \left(\frac{M}{M_*}\right)^{-\beta+1}.
\end{equation}
for rocky bodies about a typical WD.  For $\beta = 1.75$, and using a typical accretion rate of $\dot{m} = 10^{16}{\rm g/yr}$, we obtain a rate of collisions shown in Figure~\ref{fig:coll_rate}.
